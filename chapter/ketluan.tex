\setcounter{chapter}{5}
\chapter{Kết luận và hướng phát triển}

\section{Kết luận}
Trải qua khoảng thời gian đề cương luận văn và cả luận văn tốt nghiệp, chúng em đã tiếp thu cũng như học hỏi được thêm nhiều kiến thức về việc phân tích, xây dựng nên một công cụ kèm theo đó là tăng khả năng lập trình. Những kỹ năng này thật sự rất bổ ích, cần thiết cho nhóm chúng em để làm kiến thức chuẩn bị cho công việc sau này.

Về chuyên môn, nhóm chúng em đã phần nào xây dựng được công cụ nhằm giải quyết được yêu cầu đặt ra ban đầu. Sau đây là đánh giá cho từng phần trong luận văn mà nhóm đã làm được một cách cụ thể:

\begin{itemize}
	\item Phần 1: Công cụ EHAT đối với GV.
	Phần này được chia làm 4 chức năng chính, trong những chức năng đó vẫn còn có những hạn chế mà nhóm chưa cải thiện được trong quá trình luận văn này
	\begin{itemize}
		\item Chức năng đầu tiên mà nhóm kể đến đó là EHAT có thể được sử dụng để đánh giá chi tiết năng lực của sinh viên thông qua biểu đồ mạng nhện. 
		
		Tuy vậy chức năng này vẫn còn mặt hạn chế đó là việc tính toán số điểm của từng tiêu chí mỗi sinh viên chưa thật sự đa dạng chỉ dựa vào điểm trung bình các tiêu chí mà GV chọn lựa.
		
		\item Chức năng thứ hai mà nhóm làm được là biểu đồ cột ngang nhằm thống kê số lượt truy cập của sinh viên vào một hạng mục cụ thể. 
		
		Hiện ở chức năng này nhóm vẫn chưa cải thiện được giao diện cũng như dữ liệu hiện ra vẫn còn chưa tối giản.
		
		\item Biểu đồ phân phối lượt truy cập của từng cá nhân HS, SV là chức năng thứ ba mà nhóm muốn nói. Dựa vào biểu đồ này GV có thể thấy được thái độ học tập của mỗi HV trong khóa học. 
		
		Mặt hạn chế của chức năng này đó là nhóm vẫn chưa hiển thị được thời gian mà HV truy cập vào tài nguyên.
		
		\item Chức năng cuối cùng đó là bảng hỗ trợ cho GV thêm tài liệu tham khảo của mình để HV có thể thấy và tham khảo. 
		
		Tuy vậy chức năng này còn đang hoạt động một cách thủ công chưa được tự động.
	\end{itemize}
	\item Phần 2: Công cụ EHAT đối với HS, SV
	
	Đối với HS, SV bởi nhóm chưa nghĩ ra được thêm chức năng nào khác dành cho HS, SV nên hiện tại nhóm chỉ làm được một chức năng đó là bảng số liệu nhằm giúp HV tham khảo lại chi tiết bài kiểm tra cuối cùng của mình, bên cạnh đó có hiển thị thông tin nội dung tham khảo mà GV đã thêm vào.
	
	Hạn chế của chức năng này hiện vẫn còn hạn chế đó là HS chỉ được tham khảo lại chi tiết bài kiểm tra cuối cùng mà không xem được các bài kiểm tra của những lần khác, giao diện vẫn còn đơn giản chưa thật sự bắt mắt.
\end{itemize}

\section{Hướng phát triển}

Để sản phẩm được hoàn thiện hơn thì nhóm sẽ cải thiện về những mặt sau đây:

\begin{itemize}
	\item Ngoài việc tính điểm dựa trên việc lấy trung bình nhóm sẽ nghiên cứu và thêm nhiều thuật toán hơn để tính điểm của từng tiêu chí để đánh giá một cách chính xác hơn từng cá nhân sinh viên.
	
	\item Cải thiện về mặt giao diện với từng chức năng mà nhóm đã hiện thực.
	
	\item Ở phần hiển thị chi tiết số lần truy cập của sinh viên đối với một tài nguyên cụ thể nhóm sẽ tiếp tục xây dựng để hiển thị thêm khoảng thời gian mà sinh viên ấy tương tác với tài nguyên là lúc nào để GV đánh giá thói quen học tập của SV chính xác hơn.
	
	\item Cải thiện chức năng cuối cùng mà EHAT dành cho GV để việc thêm tài liệu sẽ hoạt động một cách tự động nhằm tiết kiệm thời gian cho GV.
	
	\item Nghiên cứu và xây dựng thêm những chức năng để hỗ trợ cho SV.
	
	\item Phát triển thêm để công cụ có thể tự xây dựng khóa học phù hợp dành riêng cho từng cá nhân sinh viên.
\end{itemize}

\section{Lời kết}

Trải qua khoảng thời gian hai học kỳ của đề cương và làm luận văn nhóm chúng em thật sự rất cảm kích, biết ơn đến tất cả các thầy cô đã dạy dỗ, truyền đạt những kiến thức cần thiết để chúng em có thể thực hiện tốt đề tài của thầy Thoại Nam. Bên cạnh đó luận văn cũng đã giúp chúng em có cái nhìn rộng hơn về việc phân tích, xây dựng một công cụ, một ứng dụng phục vụ cho một nhu cầu nào đó. 

Một lần nữa, chúng em xin cảm ơn thầy Thoại Nam dù biết thầy rất bận rộn với công việc của mình nhưng thầy vẫn luôn gắn bó với tụi em trong suốt giai đoạn từ đề cương cho đến luận văn. Cám ơn thầy đã cho chúng em thử thách cuối cùng thật ý nghĩa này trước khi chúng em bước ra trải nghiệm môi trường thực tế đầy những khó khăn phải đối mặt. 

Cảm ơn Bách Khoa nơi đã mang đến cho chúng em một môi trường gian khổ nhưng vui vẻ, hạnh phúc.