\setcounter{chapter}{5}
\chapter{Kết luận và hướng phát triển}

\section{Kết luận}
Trải qua khoảng thời gian đề cương luận văn và cả luận văn tốt nghiệp, nhóm đã tiếp thu cũng như học hỏi được thêm nhiều kiến thức về việc phân tích, xây dựng nên một công cụ. Những kỹ năng này thật sự rất bổ ích, cần thiết cho nhóm chúng em để làm kiến thức chuẩn bị cho công việc sau này.

Về chuyên môn, nhóm chúng em đã phần nào xây dựng được công cụ nhằm giải quyết được yêu cầu đặt ra ban đầu. Sau đây là đánh giá cho từng phần trong luận văn mà nhóm đã làm được một cách cụ thể:

\begin{itemize}
	\item Phần 1: Công cụ EHAT đối với GV.
	Phần này được chia làm năm chức năng chính, trong những chức năng đó vẫn còn có những hạn chế mà nhóm chưa cải thiện được trong quá trình luận văn này
	\begin{itemize}
		\item Chức năng đầu tiên mà nhóm kể đến đó là EHAT có thể được sử dụng để đánh giá chi tiết và so sánh năng lực của học viên thông qua biểu đồ mạng nhện. 
		
		Tuy vậy chức năng này vẫn còn mặt hạn chế đó là việc tính toán số điểm của từng tiêu chí mỗi học viên chưa thật sự đa dạng chỉ dựa vào điểm trung bình các tiêu chí mà GV chọn lựa.
		
		\item Chức năng thứ hai mà nhóm làm được là biểu đồ cột ngang nhằm thống kê số lượt truy cập của học viên vào một hạng mục cụ thể. 
		
		Hiện ở chức năng này nhóm vẫn chưa cải thiện được giao diện cũng như dữ liệu hiện ra vẫn còn chưa tối giản.
		
		\item Biểu đồ phân phối lượt truy cập của từng cá nhân HV là chức năng thứ ba mà nhóm muốn nói. Dựa vào biểu đồ này GV có thể thấy được thái độ học tập của mỗi HV trong khóa học. 
		
		Mặt hạn chế của chức năng này đó là nhóm vẫn chưa hiển thị được thời gian mà HV truy cập vào tài nguyên.
		
		\item Thứ tư biểu đồ mô tả hoạt động của học viên theo từng khung giờ nhằm giúp giảng viên phân nhóm học viên tương tác với khóa học theo giờ.
		
		Chức năng này vẫn chưa mô tả được chi tiết cụ thể ngày nào kể từ ngày bắt đầu khóa học
		
		\item Chức năng cuối cùng đó là bảng hỗ trợ cho GV thêm tài liệu tham khảo của mình để HV có thể thấy và tham khảo. 
		
		Tuy vậy chức năng này còn đang hoạt động một cách thủ công chưa được tự động.
	\end{itemize}

	\vskip 3cm
	\item Phần 2: Công cụ EHAT đối với HV
	
	\begin{itemize}
		\item Đối với HV bởi nhóm chưa nghĩ ra được thêm chức năng nào khác dành cho HV nên hiện tại nhóm chỉ làm được một chức năng đó là bảng số liệu nhằm giúp HV tham khảo lại chi tiết bài kiểm tra cuối cùng của mình, bên cạnh đó có hiển thị thông tin nội dung tham khảo mà GV đã thêm vào.
		
		Hạn chế của chức năng này hiện vẫn còn hạn chế đó là HV chỉ được tham khảo lại chi tiết bài kiểm tra cuối cùng mà không xem được các bài kiểm tra của những lần khác, giao diện vẫn còn đơn giản chưa thật sự bắt mắt.
	\end{itemize}
	
\end{itemize}

\section{Hướng phát triển}

Để sản phẩm được hoàn thiện hơn, trong tương lai nhóm sẽ cải thiện sản phẩm của mình về những mặt sau đây:

\begin{itemize}
	\item Ngoài việc tính điểm dựa trên việc lấy trung bình nhóm sẽ nghiên cứu và phát triển thêm nhiều thuật toán hơn để tính điểm của từng tiêu chí nhằm đánh giá một cách chính xác hơn từng cá nhân học viên.
	
	\item Cải thiện về mặt giao diện với từng chức năng mà nhóm đã hiện thực.
	
	\item Ở phần hiển thị chi tiết số lần truy cập của học viên đối với một tài nguyên cụ thể nhóm sẽ tiếp tục xây dựng để hiển thị thêm khoảng thời gian mà học viên ấy tương tác với tài nguyên là lúc nào để GV đánh giá thói quen học tập của HV chính xác hơn.
	
	\item Cải thiện chức năng cuối cùng mà EHAT dành cho GV để việc thêm tài liệu sẽ hoạt động một cách tự động nhằm tiết kiệm thời gian cho GV.
	
	\item Nghiên cứu và xây dựng thêm những chức năng để hỗ trợ cho HV.
	
	\item Phát triển thêm để công cụ có thể tự xây dựng khóa học phù hợp dành riêng cho từng cá nhân học viên dựa trên điểm đầu vào.
	
	\item Phát triển thêm tính năng chọn ngày cụ thể để tham khảo cho biểu đồ mô tả hoạt động của học viên theo khung giờ
\end{itemize}