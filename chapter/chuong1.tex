\setcounter{chapter}{0}
\fontsize{13}{5}\selectfont
\chapter{Tổng quan về đề tài}
\section{Giới thiệu đề tài}
Trong thời đại công nghệ 4.0 hiện nay, việc học tập trực tuyến đối với mọi người càng trở nên phổ biến. Đã có rất nhiều trang đào tạo trực tuyến ở Việt Nam phục vụ cho việc học tập trực tuyến. Trong quá trình tìm hiểu về những trang đào tạo ấy chúng em đã rút ra được những vấn đề mà mình cần phải giải quyết trong phạm vi luận này đó là:

\begin{itemize}
	\item Người học có thể tận dụng khoảng thời gian tối thiểu để thu về được lượng kiến thức tối đa.
	\item Giáo viên có thể đánh giá được chi tiết từng cá nhân học viên trong học.
	\item Giáo viên đánh giá được mức độ hiệu quả của việc xây dựng bài giảng, bài kiểm tra trong khóa học của mình.
	\item Mô tả được thái độ cũng như là hành vi học tập của từng cá nhân học viên trong khóa học.
\end{itemize}

Để giải quyết những vấn đề trên nhóm đã quyết định xây dựng một công cụ chạy trên nền tảng Moodle có tên là E-learning Help And Assess Tool(EHAT) có những chức năng sau:

\begin{itemize}
	\item Về khóa học mẫu:
	\begin{itemize}
		\item Nội dung bài giảng ngắn gọn, dễ hiểu không gây nhàm chán.
		\item Có video giảng dạy.
		\item Có một hoặc nhiều bài kiểm tra đánh giá sau mỗi chương trong mỗi khóa học.
		\item Có bài kiểm tra tổng hợp
	\end{itemize}
	\item Về công cụ hỗ trợ đánh giá (EHAT):
	\begin{itemize}
		\item Biểu đồ radar đánh giá chi tiết điểm của từng sinh viên trong mỗi khóa học.
		\item So sánh biểu đồ của hai sinh viên.
		\item Biểu đồ thống kê số lượt truy cập của sinh viên
		\item Biểu đồ phân phối lượt truy cập của từng sinh viên trong khóa học
		\item Bảng hỗ trợ giảng viên thêm tài liệu tham khảo
	\end{itemize}
\end{itemize}

\section{Lý do chọn đề tài}
Chắc hẳn trong chúng ta ai cũng đã từng trải qua giai đoạn ngồi trên ghế nhà trường và cảm nhận được những khó khăn nhất định trong quá trình học tập của mình như:
\begin{itemize}
	\item Không linh hoạt về thời gian.
	\item Tốn kém hơn về chi phí và công sức.
	\item Khó có lại kiến thức nếu vắng một buổi học.
	\item Sự tương tác giữa học sinh, sinh viên với giáo viên thấp.
	\item Đánh giá kết quả thông qua các bài kiểm tra.
	\item Không đánh giá được thái độ học tập của từng sinh viên trong quá trình học tập trực tuyến
\end{itemize}
Nắm bắt được những khó khăn đó cùng với niềm mong muốn tạo ra một khóa học trực tuyến nhằm để tạo điều kiện thuận lợi hơn trong việc giảng dạy cũng như đánh giá năng lực của học viên. Nhóm chúng em đã lựa chọn đề tài này làm đề tài để nghiên cứu và thực hiện.

\section{Mục tiêu của đề tài}
\begin{itemize}
	\item Xây dựng thành công công cụ EHAT trên nền tảng Moodle để có thể đánh giá cũng như hỗ trợ được học viên trong suốt quá trình học tập trực tuyến.
	\item Công cụ tương thích với hầu hết các khóa học sử dụng nền tảng Moodle
	\item EHAT dễ dàng nâng cấp cũng như mở rộng trong tương lai.
\end{itemize}

\section{Phương pháp hiện thực đề tài}
Từ những mục tiêu đề ra cùng với việc sử dụng nền tảng Moodle để xây dựng công cụ. Nhóm chúng em quyết định xây dựng EHAT bằng ngôn ngữ lập trình PHP, MySQL và Apache làm server.

\section{Cấu trúc luận văn}
Bố cục của luận văn bao gồm các chương sau:
\begin{itemize}
	\item Chương 1: Tổng quan về đề tài. Chương này sẽ giới thiệu đề tài, lý do chọn đề tài, mục tiêu và phương pháp của đề tài.
	\item Chương 2: Chương này sẽ trình bày tóm tắt những kiến thức nền tảng liên quan đến quá trình xây dựng hệ thống.
	\item Chương 3: Phân tích hệ thống use case, database, những dạng biểu đồ sử dụng trong công cụ.
	\item Chương 4: Hiện thực và triển khai.
	\item Chương 5: Kết luận và hướng phát triển đề tài.
\end{itemize}