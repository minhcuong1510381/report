\fontsize{13}{5}\selectfont
\setcounter{chapter}{0}
\chapter{Tổng quan về đề tài}
\section{Giới thiệu đề tài}
Đề tài tập trung xây dựng một công cụ có tên là E-learning Help And Assess Tool(EHAT) nhằm giúp cho giảng viên có thể đánh giá một cách cụ thể kết quả học tập của học viên trong mỗi khóa học. Để làm được như thế, khóa học cũng như EHAT sẽ thỏa những tiêu chí sau:
\begin{itemize}
	\item Về khóa học mẫu:
	\begin{itemize}
		\item Nội dung bài giảng ngắn gọn, dễ hiểu không gây nhàm chán.
		\item Có video giảng dạy.
		\item Có một hoặc nhiều bài kiểm tra đánh giá sau mỗi chương trong mỗi khóa học.
	\end{itemize}
	\item Về công cụ hỗ trợ đánh giá (EHAT):
	\begin{itemize}
		\item Biểu đồ radar \& bảng số liệu đánh giá chi tiết điểm của sinh viên trong mỗi khóa học.
		\item So sánh biểu đồ của hai sinh viên.
		\item Biểu đồ tổng kết điểm trung bình của tất cả học viên trong khóa học.
		\item Biểu đồ thống kê phần trăm sinh viên trên trung bình của từng chương trong khóa học.
		\item Bảng đánh giá chi tiết kết quả trong lần kiểm tra cuối cùng của sinh viên.
	\end{itemize}
\end{itemize}

\section{Lý do chọn đề tài}
Chắc hẳn trong chúng ta ai cũng đã từng trải qua giai đoạn ngồi trên ghế nhà trường và cảm nhận được những khó khăn nhất định trong quá trình học tập của mình như:
\begin{itemize}
	\item Không linh hoạt về thời gian.
	\item Tốn kém hơn về chi phí và công sức.
	\item Khó có lại kiến thức nếu vắng một buổi học.
	\item Sự tương tác giữa học sinh, sinh viên với giáo viên thấp.
	\item Đánh giá kết quả thông qua các bài kiểm tra.
\end{itemize}
Nắm bắt được những khó khăn đó cùng với niềm mong muốn tạo ra một khóa học trực tuyến nhằm để tạo điều kiện thuận lợi hơn trong việc giảng dạy cũng như đánh giá năng lực của học viên. Nhóm chúng em đã lựa chọn đề tài này làm đề tài để nghiên cứu và thực hiện.

\section{Mục tiêu của đề tài}
\begin{itemize}
	\item Xây dựng thành công công cụ EHAT trên nền tảng Moodle để có thể đánh giá cũng như hỗ trợ được học viên trong suốt quá trình học tập trực tuyến.
	\item Công cụ tương thích với hầu hết các khóa học sử dụng nền tảng Moodle
	\item EHAT dễ dàng nâng cấp cũng như mở rộng trong tương lai.
\end{itemize}

\section{Phương pháp hiện thực đề tài}
Từ những mục tiêu đề ra cùng với việc sử dụng nền tảng Moodle để xây dựng công cụ. Nhóm chúng em quyết định xây dựng EHAT bằng ngôn ngữ lập trình PHP, MySQL và Apache làm server.